%------------------------------------------------------------------------------------
%	PACKAGES AND OTHER DOCUMENT CONFIGURATIONS
%------------------------------------------------------------------------------------

\documentclass{article}

\usepackage{fancyhdr} % Required for custom headers
\usepackage{lastpage} % Required to determine the last page for the footer
\usepackage{extramarks} % Required for headers and footers
\usepackage[usenames,dvipsnames]{color} % Required for custom colors
\usepackage{graphicx} % Required to insert images
\usepackage{subcaption}
\usepackage{listings} % Required for insertion of code
\usepackage{courier} % Required for the courier font
% Optional Packages
\usepackage{amsmath}
\usepackage{amssymb}
\usepackage{float}
\usepackage{algorithm}
\usepackage[noend]{algpseudocode}


% Margins
\topmargin=-0.45in
\evensidemargin=0in
\oddsidemargin=0in
\textwidth=6.5in
\textheight=9.0in
\headsep=0.25in

\linespread{1.1} % Line spacing

% Set up the header and footer
\pagestyle{fancy}
\lhead{\hmwkAuthorName} % Top left header
\chead{\hmwkClass\ : \hmwkTitle} % Top center head
%\rhead{\firstxmark} % Top right header
\lfoot{\lastxmark} % Bottom left footer
\cfoot{} % Bottom center footer
\rfoot{Page\ \thepage\ of\ \protect\pageref{LastPage}} % Bottom right footer
\renewcommand\headrulewidth{0.4pt} % Size of the header rule
\renewcommand\footrulewidth{0.4pt} % Size of the footer rule

\setlength\parindent{0pt} % Removes all indentation from paragraphs


%------------------------------------------------------------------------------------
%	DOCUMENT STRUCTURE COMMANDS
%	Skip this unless you know what you're doing
%------------------------------------------------------------------------------------

% Header and footer for when a page split occurs within a problem environment
\newcommand{\enterProblemHeader}[1]{
	%\nobreak\extramarks{#1}{#1 continued on next page\ldots}\nobreak
	%\nobreak\extramarks{#1 (continued)}{#1 continued on next page\ldots}\nobreak
}

% Header and footer for when a page split occurs between problem environments
\newcommand{\exitProblemHeader}[1]{
	%\nobreak\extramarks{#1 (continued)}{#1 continued on next page\ldots}\nobreak
	%\nobreak\extramarks{#1}{}\nobreak
}

\setcounter{secnumdepth}{0} % Removes default section numbers
\newcounter{homeworkProblemCounter} % Creates a counter to keep track of the number of problems
\setcounter{homeworkProblemCounter}{0}

\newcommand{\homeworkProblemName}{}
\newenvironment{homeworkProblem}[1][Problem \arabic{homeworkProblemCounter}]{ % Makes a new environment called homeworkProblem which takes 1 argument (custom name) but the default is "Problem #"
	\stepcounter{homeworkProblemCounter} % Increase counter for number of problems
	\renewcommand{\homeworkProblemName}{#1} % Assign \homeworkProblemName the name of the problem
	\section{\homeworkProblemName} % Make a section in the document with the custom problem count
	\enterProblemHeader{\homeworkProblemName} % Header and footer within the environment
}{
	\exitProblemHeader{\homeworkProblemName} % Header and footer after the environment
}

\newcommand{\problemAnswer}[1]{ % Defines the problem answer command with the content as the only argument
	\noindent\framebox[\columnwidth][c]{\begin{minipage}{0.98\columnwidth}#1\end{minipage}} % Makes the box around the problem answer and puts the content inside
}

\newcommand{\homeworkSectionName}{}
\newenvironment{homeworkSection}[1]{ % New environment for sections within homework problems, takes 1 argument - the name of the section
	\renewcommand{\homeworkSectionName}{#1} % Assign \homeworkSectionName to the name of the section from the environment argument
	\subsection{\homeworkSectionName} % Make a subsection with the custom name of the subsection
	\enterProblemHeader{\homeworkProblemName\ [\homeworkSectionName]} % Header and footer within the environment
}{
	\enterProblemHeader{\homeworkProblemName} % Header and footer after the environment
}


%=================================================================

%------------------------------------------------------------------------------------
%	NAME AND CLASS SECTION
%------------------------------------------------------------------------------------

\newcommand{\hmwkTitle}{Assignment\ \#4} % Assignment title
\newcommand{\hmwkClass}{CSC321} % Course/class
\newcommand{\hmwkAuthorName}{Xiangyu Kong} % Your name
\newcommand{\hmwkUTorId}{kongxi16} % UTorID

%------------------------------------------------------------------------------------
%	TITLE PAGE
%------------------------------------------------------------------------------------

\title{
	\vspace{2in}
	\textmd{\textbf{\hmwkClass:\ \hmwkTitle}}\\
	%	\normalsize\vspace{0.1in}\small{Due\ on\ \hmwkDueDate}\\
	\vspace{0.1in}
	\vspace{3in}
}

\author{\textbf{\hmwkAuthorName} \\ \textbf{\hmwkUTorId}}

% Insert date here if you want it to appear below your name
\date{\today} 

%------------------------------------------------------------------------------------

\begin{document}
	
	\maketitle
	\clearpage
	
	%---------------------------------------------------------------------------------
	%	PROBLEM 1
	%---------------------------------------------------------------------------------
	\begin{homeworkProblem}
		
		\begin{enumerate}
			\item \large{\textbf{Implement the Discriminator of the DCGAN}}
			\begin{enumerate}
				\item 
				Let $P = \text{number of padding}$, $W = \text{input width}$, $K = \text{kernel size}$, $S = \text{stride}$, $O = \text{output width}$
				
				According to the structure of the convolutional network, 
				\begin{align*}
					O &= \dfrac{W}{2}
				\end{align*}
				
				Using the formula for calculating output dimension
				\begin{align*}
					O &= \frac{W - K + 2P}{S} \\
					P &= \frac{S (\frac{W}{2} - 1) + K - W}{2} \\
						&= \frac{2 (\frac{W}{2} - 1) + 4 - W}{2} \\
						&= \frac{W - 2 + 4 - W}{2} \\
						&= 1
				\end{align*}
				
				\item
				See implementation in models.py
			\end{enumerate}
			
			%-----------------------------------------------------------------------------
			
			\item \large{\textbf{Generator}}\\
			See implementation in models.py
			
			\item \large{\textbf{Experiment}}\\
			Samples from iterations 200 and 5000 are shown in Fig\ref{fig:1}.
			The sample improves in their resolution and quality. We can clearly see what each emoji is (human figure or objects) after a very long period of iterations.
			
			
			\begin{figure*}[!ht]
				\centering
				\begin{subfigure}{.45\textwidth}
					\centering
					\includegraphics[width=.8\linewidth]{images/samples_vanilla/sample-000200.png}
					\caption{Sample from iteration 200}
					\label{fig:1.1}
				\end{subfigure}
				\begin{subfigure}{.45\textwidth}
					\centering
					\includegraphics[width=.8\linewidth]{images/samples_vanilla/sample-005000.png}
					\caption{Sample from iteration 5000}
					\label{fig:1.2}
				\end{subfigure}
				\caption{}
				\label{fig:1}
			\end{figure*}

		\end{enumerate}
		
	\end{homeworkProblem}
	\clearpage
	
	%----------------------------------------------------------------------------------
	
	%---------------------------------------------------------------------------------
	%	PROBLEM 2
	%---------------------------------------------------------------------------------
	\begin{homeworkProblem}

		\begin{enumerate}
			\item \large{\textbf{Generator}}\\
			See implementation in model.py
			
			\item \large{\textbf{CycleGAN Training Loop}}\\
			See implementation in cycle\_gan.py
			
			\item \large{\textbf{CycleGAN Experiments}}
			\begin{enumerate}
				\item % 1
				The samples without cycle-consistency loss at iteration 400 are shown in Fig\ref{fig:2}
				
				\begin{figure*}[!ht]
					\centering
					\begin{subfigure}{.45\textwidth}
						\centering
						\includegraphics[width=.8\linewidth]{images/samples_cyclegan/sample-000400-X-Y.png}
						\caption{Sample from iteration 400 X-Y}
						\label{fig:2.1}
					\end{subfigure}
					\begin{subfigure}{.45\textwidth}
						\centering
						\includegraphics[width=.8\linewidth]{images/samples_cyclegan/sample-000400-Y-X.png}
						\caption{Sample from iteration 400 Y-X}
						\label{fig:2.2}
					\end{subfigure}
					\caption{}
					\label{fig:2}
				\end{figure*}
				
				\item % 2
				The samples with cycle-consistency loss at iteration 400 are shown in Fig\ref{fig:3}
				
				\begin{figure*}[!ht]
					\centering
					\begin{subfigure}{.45\textwidth}
						\centering
						\includegraphics[width=.8\linewidth]{images/samples_cyclegan_cycle/sample-000400-X-Y.png}
						\caption{Sample from iteration 400 X-Y}
						\label{fig:3.1}
					\end{subfigure}
					\begin{subfigure}{.45\textwidth}
						\centering
						\includegraphics[width=.8\linewidth]{images/samples_cyclegan_cycle/sample-000400-Y-X.png}
						\caption{Sample from iteration 400 Y-X}
						\label{fig:3.2}
					\end{subfigure}
					\caption{}
					\label{fig:3}
				\end{figure*}
				
				\item % 3
				The samples from the pre-trained network without cycle-consistency loss are shown in Fig\ref{fig:4}
				
				\begin{figure*}[!ht]
					\centering
					\begin{subfigure}{.45\textwidth}
						\centering
						\includegraphics[width=.8\linewidth]{images/samples_cyclegan_pretrained/sample-000100-X-Y.png}
						\caption{Sample from iteration 100 X-Y}
						\label{fig:4.1}
					\end{subfigure}
					\begin{subfigure}{.45\textwidth}
						\centering
						\includegraphics[width=.8\linewidth]{images/samples_cyclegan_pretrained/sample-000100-Y-X.png}
						\caption{Sample from iteration 100 Y-X}
						\label{fig:4.2}
					\end{subfigure}
					\caption{}
					\label{fig:4}
				\end{figure*}
				
				\item % 4
				The samples from the pre-trained network with cycle-consistency loss are shown in Fig\ref{fig:5}
				
				\begin{figure*}[!ht]
					\centering
					\begin{subfigure}{.45\textwidth}
						\centering
						\includegraphics[width=.8\linewidth]{images/samples_cyclegan_cycle_pretrained/sample-000100-X-Y.png}
						\caption{Sample from iteration 100 X-Y}
						\label{fig:5.1}
					\end{subfigure}
					\begin{subfigure}{.45\textwidth}
						\centering
						\includegraphics[width=.8\linewidth]{images/samples_cyclegan_cycle_pretrained/sample-000100-Y-X.png}
						\caption{Sample from iteration 100 Y-X}
						\label{fig:5.2}
					\end{subfigure}
					\caption{}
					\label{fig:5}
					\end{figure*}
				
				\item % 5
				By observing the samples of the same iteration but different loss (with and without cycle consistency loss), it is obvious that the models using cycle consistency loss generate similar and clearer emojis. This is because by using the cycle consistency loss, we make the network account for the reconstruction loss. Then the generators will try to generate emojis with style more similar to the target. Thus the samples will get a better resemblance and become clearer. 
			\end{enumerate}
		\end{enumerate}
				
	\end{homeworkProblem}
	\clearpage
	
	%----------------------------------------------------------------------------------
	
\end{document}