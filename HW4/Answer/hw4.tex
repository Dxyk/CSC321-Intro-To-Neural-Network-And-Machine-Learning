%----------------------------------------------------------------------------------------
%	PACKAGES AND OTHER DOCUMENT CONFIGURATIONS
%----------------------------------------------------------------------------------------

\documentclass{article}

\usepackage{fancyhdr} % Required for custom headers
\usepackage{lastpage} % Required to determine the last page for the footer
\usepackage{extramarks} % Required for headers and footers
\usepackage[usenames,dvipsnames]{color} % Required for custom colors
\usepackage{graphicx} % Required to insert images
\usepackage{subcaption}
\usepackage{listings} % Required for insertion of code
\usepackage{courier} % Required for the courier font
\usepackage{lipsum} % Used for inserting dummy 'Lorem ipsum' text into the template
\usepackage{amsmath}
\usepackage{amssymb}
\usepackage{float}

% Margins
\topmargin=-0.45in
\evensidemargin=0in
\oddsidemargin=0in
\textwidth=6.5in
\textheight=9.0in
\headsep=0.25in

\linespread{1.1} % Line spacing

% Set up the header and footer
\pagestyle{fancy}
\lhead{\hmwkAuthorName} % Top left header
\chead{\hmwkClass\ : \hmwkTitle} % Top center head
%\rhead{\firstxmark} % Top right header
\lfoot{\lastxmark} % Bottom left footer
\cfoot{} % Bottom center footer
\rfoot{Page\ \thepage\ of\ \protect\pageref{LastPage}} % Bottom right footer
\renewcommand\headrulewidth{0.4pt} % Size of the header rule
\renewcommand\footrulewidth{0.4pt} % Size of the footer rule

\setlength\parindent{0pt} % Removes all indentation from paragraphs


%----------------------------------------------------------------------------------------
%	DOCUMENT STRUCTURE COMMANDS
%	Skip this unless you know what you're doing
%----------------------------------------------------------------------------------------

% Header and footer for when a page split occurs within a problem environment
\newcommand{\enterProblemHeader}[1]{
	%\nobreak\extramarks{#1}{#1 continued on next page\ldots}\nobreak
	%\nobreak\extramarks{#1 (continued)}{#1 continued on next page\ldots}\nobreak
}

% Header and footer for when a page split occurs between problem environments
\newcommand{\exitProblemHeader}[1]{
	%\nobreak\extramarks{#1 (continued)}{#1 continued on next page\ldots}\nobreak
	%\nobreak\extramarks{#1}{}\nobreak
}

\setcounter{secnumdepth}{0} % Removes default section numbers
\newcounter{homeworkProblemCounter} % Creates a counter to keep track of the number of problems
\setcounter{homeworkProblemCounter}{0}

\newcommand{\homeworkProblemName}{}
\newenvironment{homeworkProblem}[1][Problem \arabic{homeworkProblemCounter}]{ % Makes a new environment called homeworkProblem which takes 1 argument (custom name) but the default is "Problem #"
	\stepcounter{homeworkProblemCounter} % Increase counter for number of problems
	\renewcommand{\homeworkProblemName}{#1} % Assign \homeworkProblemName the name of the problem
	\section{\homeworkProblemName} % Make a section in the document with the custom problem count
	\enterProblemHeader{\homeworkProblemName} % Header and footer within the environment
}{
	\exitProblemHeader{\homeworkProblemName} % Header and footer after the environment
}

\newcommand{\problemAnswer}[1]{ % Defines the problem answer command with the content as the only argument
	\noindent\framebox[\columnwidth][c]{\begin{minipage}{0.98\columnwidth}#1\end{minipage}} % Makes the box around the problem answer and puts the content inside
}

\newcommand{\homeworkSectionName}{}
\newenvironment{homeworkSection}[1]{ % New environment for sections within homework problems, takes 1 argument - the name of the section
	\renewcommand{\homeworkSectionName}{#1} % Assign \homeworkSectionName to the name of the section from the environment argument
	\subsection{\homeworkSectionName} % Make a subsection with the custom name of the subsection
	\enterProblemHeader{\homeworkProblemName\ [\homeworkSectionName]} % Header and footer within the environment
}{
	\enterProblemHeader{\homeworkProblemName} % Header and footer after the environment
}


%========================================================================================================
%----------------------------------------------------------------------------------------
%	NAME AND CLASS SECTION
%----------------------------------------------------------------------------------------

\newcommand{\hmwkTitle}{Assignment\ \#4} % Assignment title
\newcommand{\hmwkClass}{CSC321} % Course/class
\newcommand{\hmwkAuthorName}{Xiangyu Kong} % Your name
\newcommand{\hmwkUTorId}{kongxi16} % UTorID

%----------------------------------------------------------------------------------------
%	TITLE PAGE
%----------------------------------------------------------------------------------------

\title{
	\vspace{2in}
	\textmd{\textbf{\hmwkClass:\ \hmwkTitle}}\\
	%	\normalsize\vspace{0.1in}\small{Due\ on\ \hmwkDueDate}\\
	\vspace{0.1in}
	\vspace{3in}
}

\author{\textbf{\hmwkAuthorName} \\ \textbf{\hmwkUTorId}}

% Insert date here if you want it to appear below your name
\date{\today} 

%----------------------------------------------------------------------------------------

\begin{document}
	
	\maketitle
	\clearpage
	
	
	%---------------------------------------------------------------------------------
	%	PROBLEM 1
	%---------------------------------------------------------------------------------
	
	\begin{homeworkProblem}
	
		\begin{enumerate}
			
			\item 
			\begin{align*}
			\dfrac{\partial C}{\partial \theta_i} = a_i ( \theta_i - r_i )
			\end{align*}
			then
			\begin{align*}
			\theta_i^{(t + 1)} &= \theta_i^{(t)} - \alpha \dfrac{\partial C}{\partial \theta_i}\\
			&= \theta_i^{(t)} - \alpha a_i ( \theta_i^{(t)} - r_i )
			\end{align*}
			
			\item 
			\begin{align*}
			e_i^{(t + 1)} &= \theta_i^{(t + 1)} - r_i\\
			&= \theta_i^{(t)} - \alpha a_i ( \theta_i^{(t)} - r_i ) - r_i\\
			&= e_i^{(t)} - \alpha a_i ( \theta_i^{(t)} - r_i )\\
			&= e_i^{(t)} - \alpha a_i  e_i^{(t)} 
			\end{align*}
			
			\item 
			solving the equation, 
			\begin{align*}
			e_i^{(t)} = e_i^{(0)} (1 - \alpha a_i)^{t}
			\end{align*}
			For $0 < \alpha < \dfrac{2}{a_i}$, $e_i^{(t)}$ will converge, so $e_i^{(t)}$ will be stable.\\
			For $\alpha < 0$ or $\alpha > \dfrac{2}{a_i}$, $e_i^{(t)}$ will diverge and become unstable.
			
			\item 
			\begin{align*}
			\mathcal{C}(\mathbf{\theta}^{(t)}) &= \sum \limits^{N}_{i = 0} \dfrac{a_i}{2} (e_i^{(t)})^2\\
			&= \sum \limits^{N}_{i = 0} \dfrac{a_i}{2} (e_i^{(0)} (1 - \alpha a_i)^{t})^2
			\end{align*}
			As $t \rightarrow \infty$, $a_i$ will dominate
			
		\end{enumerate}
		
	\end{homeworkProblem}
	\clearpage
	
	
	
	%---------------------------------------------------------------------------------
	%	PROBLEM 2
	%---------------------------------------------------------------------------------
	
	\begin{homeworkProblem}
		\begin{enumerate}
			\item 
			\begin{align*}
			\mathbb{E}[y] &= \mathbb{E}[\sum \limits_{j} m_j w_j x_j]\\
			&= \sum \limits_{j} \mathbb{E}[m_j w_j x_j]\\
			&= \sum \limits_{j}  w_j x_j \mathbb{E}[m_j]\\
			&= \sum \limits_{j}  \dfrac{1}{2} w_j x_j \\
			&= \dfrac{1}{2} \mathbf{w}^{\top} \mathbf{x}
			\end{align*}
			
			\begin{align*}
			Var[y] &= Var[\sum \limits_{j} m_j w_j x_j]\\
			&= \sum \limits_{j} Var[ m_j w_j x_j ] + \sum \limits_{i \neq j}Cov[w_i, w_j]\\
			&= \sum \limits_{j} (w_j x_j)^2 Var[ m_j ] \\
			&= \sum \limits_{j} \dfrac{1}{4} (w_j x_j)^2\\
			&= \sum \limits_{j} ( \dfrac{1}{2} w_j x_j)^2\\
			&= \dfrac{1}{4} (\mathbf{w}^{\top} \mathbf{x})^2
			\end{align*}
			
			\item 
			\begin{align*}
			\mathbb{E}[y] &= \sum \limits_{j}  \dfrac{1}{2} w_j x_j \\
			&= \sum \limits_{j} \tilde{w_j} x_j
			\end{align*}
			then 
			
			\begin{align*}
			\tilde{w_j} = \dfrac{1}{2} w_j
			\end{align*}
			
			\newpage
			\item
			\begin{align*}
			\mathcal{E} &= \dfrac{1}{2N} \sum\limits_{i = 1}^{N} \mathbb{E}[ (y^{(i)} - t_{(i)})^2 ]\\
			&= \dfrac{1}{2N} \sum\limits_{i = 1}^{N} \mathbb{E}[ {y^{(i)}}^2 - 2 y^{(i)} t^{(i)} + {t^{(i)}}^2 ]\\
			&= \dfrac{1}{2N} \sum\limits_{i = 1}^{N} [ \mathbb{E}[ {y^{(i)}}^2 ] - \mathbb{E} [2 y^{(i)} t^{(i)}] + \mathbb{E} [{t^{(i)}}^2 ] ]\\
			&= \dfrac{1}{2N} \sum\limits_{i = 1}^{N} [
			 	\mathbb{E}[y^{(i)}] ^2 + Var[y^{(i)}] - 2 \mathbb{E} [ y^{(i)} ] \mathbb{E} [ t^{(i)} ] + \mathbb{E}[t^{(i)}] ^2 + Var[t^{(i)}] 
			 ]\\
			 &= \dfrac{1}{2N} \sum\limits_{i = 1}^{N} [
			 	(\sum \limits_{j}  \dfrac{1}{2} w_j x_j^{(i)}) ^2 + 
			 	( \sum \limits_{j} (\dfrac{1}{2} w_j x_j^{(i)})^2) - 
			 	t^{(i)} (\sum \limits_{j} w_j x_j^{(i)}) + 
			 	{t^{(i)}}^2 
			 ]
			\end{align*}
			substituting $\tilde{w}_j = \dfrac{1}{2} w_j$,
			\begin{align*}
			\mathcal{E}  &= \dfrac{1}{2N} \sum\limits_{i = 1}^{N} [
				(\sum \limits_{j}  \tilde{w}_j x_j^{(i)}) ^2 + 
				( \sum \limits_{j} (\tilde{w}_j x_j^{(i)})^2) - 
				2 t^{(i)} (\sum \limits_{j} \tilde{w}_j x_j^{(i)}) + 
				{t^{(i)}}^2
			]
			\end{align*}
			substituting $\tilde{y} = \sum \limits_{j} \tilde{w}_j x_j$,
			\begin{align*}
			\mathcal{E}  &= \dfrac{1}{2N} \sum\limits_{i = 1}^{N} [
				\tilde{y}^{(i)2} + 
				( \sum \limits_{j} (\tilde{w}_j x_j^{(i)})^2) - 
				2 t^{(i)} \tilde{y}^{(i)} + 
				{t^{(i)}}^2
			]\\
			&= \dfrac{1}{2N} \sum\limits_{i = 1}^{N} (\tilde{y}^{(i)} - t^{(i)})^2 + \dfrac{1}{2N} \sum\limits_{i = 1}^{N} ( \sum \limits_{j} (\tilde{w}_j x_j^{(i)})^2)
			\end{align*}
			Then $\mathcal{R} = \sum\limits_{i = 1}^{N} ( \sum \limits_{j} (\tilde{w}_j x_j^{(i)})^2)$
			
		\end{enumerate}
		
	\end{homeworkProblem}
	\clearpage
	
	%----------------------------------------------------------------------------------------
	
\end{document}