%----------------------------------------------------------------------------------------
%	PACKAGES AND OTHER DOCUMENT CONFIGURATIONS
%----------------------------------------------------------------------------------------

\documentclass{article}

\usepackage{fancyhdr} % Required for custom headers
\usepackage{lastpage} % Required to determine the last page for the footer
\usepackage{extramarks} % Required for headers and footers
\usepackage[usenames,dvipsnames]{color} % Required for custom colors
\usepackage{graphicx} % Required to insert images
\usepackage{subcaption}
\usepackage{listings} % Required for insertion of code
\usepackage{courier} % Required for the courier font
\usepackage{lipsum} % Used for inserting dummy 'Lorem ipsum' text into the template
\usepackage{amsmath}
\usepackage{amssymb}
\usepackage{float}

% Margins
\topmargin=-0.45in
\evensidemargin=0in
\oddsidemargin=0in
\textwidth=6.5in
\textheight=9.0in
\headsep=0.25in

\linespread{1.1} % Line spacing

% Set up the header and footer
\pagestyle{fancy}
\lhead{\hmwkAuthorName} % Top left header
\chead{\hmwkClass\ : \hmwkTitle} % Top center head
%\rhead{\firstxmark} % Top right header
\lfoot{\lastxmark} % Bottom left footer
\cfoot{} % Bottom center footer
\rfoot{Page\ \thepage\ of\ \protect\pageref{LastPage}} % Bottom right footer
\renewcommand\headrulewidth{0.4pt} % Size of the header rule
\renewcommand\footrulewidth{0.4pt} % Size of the footer rule

\setlength\parindent{0pt} % Removes all indentation from paragraphs


%----------------------------------------------------------------------------------------
%	DOCUMENT STRUCTURE COMMANDS
%	Skip this unless you know what you're doing
%----------------------------------------------------------------------------------------

% Header and footer for when a page split occurs within a problem environment
\newcommand{\enterProblemHeader}[1]{
	%\nobreak\extramarks{#1}{#1 continued on next page\ldots}\nobreak
	%\nobreak\extramarks{#1 (continued)}{#1 continued on next page\ldots}\nobreak
}

% Header and footer for when a page split occurs between problem environments
\newcommand{\exitProblemHeader}[1]{
	%\nobreak\extramarks{#1 (continued)}{#1 continued on next page\ldots}\nobreak
	%\nobreak\extramarks{#1}{}\nobreak
}

\setcounter{secnumdepth}{0} % Removes default section numbers
\newcounter{homeworkProblemCounter} % Creates a counter to keep track of the number of problems
\setcounter{homeworkProblemCounter}{0}

\newcommand{\homeworkProblemName}{}
\newenvironment{homeworkProblem}[1][Problem \arabic{homeworkProblemCounter}]{ % Makes a new environment called homeworkProblem which takes 1 argument (custom name) but the default is "Problem #"
	\stepcounter{homeworkProblemCounter} % Increase counter for number of problems
	\renewcommand{\homeworkProblemName}{#1} % Assign \homeworkProblemName the name of the problem
	\section{\homeworkProblemName} % Make a section in the document with the custom problem count
	\enterProblemHeader{\homeworkProblemName} % Header and footer within the environment
}{
	\exitProblemHeader{\homeworkProblemName} % Header and footer after the environment
}

\newcommand{\problemAnswer}[1]{ % Defines the problem answer command with the content as the only argument
	\noindent\framebox[\columnwidth][c]{\begin{minipage}{0.98\columnwidth}#1\end{minipage}} % Makes the box around the problem answer and puts the content inside
}

\newcommand{\homeworkSectionName}{}
\newenvironment{homeworkSection}[1]{ % New environment for sections within homework problems, takes 1 argument - the name of the section
	\renewcommand{\homeworkSectionName}{#1} % Assign \homeworkSectionName to the name of the section from the environment argument
	\subsection{\homeworkSectionName} % Make a subsection with the custom name of the subsection
	\enterProblemHeader{\homeworkProblemName\ [\homeworkSectionName]} % Header and footer within the environment
}{
	\enterProblemHeader{\homeworkProblemName} % Header and footer after the environment
}


%========================================================================================================
%----------------------------------------------------------------------------------------
%	NAME AND CLASS SECTION
%----------------------------------------------------------------------------------------

\newcommand{\hmwkTitle}{Assignment\ \#5} % Assignment title
\newcommand{\hmwkClass}{CSC321} % Course/class
\newcommand{\hmwkAuthorName}{Xiangyu Kong} % Your name
\newcommand{\hmwkUTorId}{kongxi16} % UTorID

%----------------------------------------------------------------------------------------
%	TITLE PAGE
%----------------------------------------------------------------------------------------

\title{
	\vspace{2in}
	\textmd{\textbf{\hmwkClass:\ \hmwkTitle}}\\
	%	\normalsize\vspace{0.1in}\small{Due\ on\ \hmwkDueDate}\\
	\vspace{0.1in}
	\vspace{3in}
}

\author{\textbf{\hmwkAuthorName} \\ \textbf{\hmwkUTorId}}

% Insert date here if you want it to appear below your name
\date{\today} 

%----------------------------------------------------------------------------------------

\begin{document}
	
	\maketitle
	\clearpage
	
	
	%---------------------------------------------------------------------------------
	%	PROBLEM 1
	%---------------------------------------------------------------------------------
	
	\begin{homeworkProblem}
		
		Let $\mathbf{x}^{ (t) }$ be a $2 \times 1$ vector containing $x_1$ and $x_2$ as the binary input at time $t$
		
		Let $\mathbf{h}^{ (t) }$ be a $3 \times 1$ vector containing $h_1$, $h_2$ and $h_3$ at time $t$ as hinted in the handout.
		
		Let $y^{ (t) }$ be a scaler of the output binary digit.\\
		
		We let 
		\begin{align*}
			% U
			\mathbf{U} &= 
			\begin{Bmatrix}
			1 & 1 \\
			1 & 1 \\
			1 & 1 \\
			\end{Bmatrix}\\
			% W
			\mathbf{W} &= 
			\begin{Bmatrix}
			1 & 1 & 1 \\
			1 & 1 & 1 \\
			1 & 1 & 1 \\
			\end{Bmatrix}\\
			% b_h
			\mathbf{b_h} &= 
			\begin{Bmatrix}
			-0.5 \\
			-1.5 \\
			-2.5 \\
			\end{Bmatrix}\\
			% v
			\mathbf{v} &= 
			\begin{Bmatrix}
			1 & -1 & 1 \\
			\end{Bmatrix}\\
			% b_y
			b_y &= -0.5
		\end{align*}
		
		Then for all $t \geq 1$,
		\begin{align}
			\mathbf{h}^{ (t) } &= \mathbf{U} \mathbf{x}^{ (t) } + \mathbf{W} \mathbf{h}^{ (t - 1) } + \mathbf{b_h}\\
			y^{ (t) } &= \mathbf{v} \mathbf{h}^{ (t) } + b_y
		\end{align}
		
		Expanding Equation (1), we get
		\begin{align*}
			h^{ (t) }_{1} &= x^{ (t) }_1 + x^{ (t) }_2 + h^{ (t - 1) }_1 - 0.5 \\
			h^{ (t) }_{2} &= x^{ (t) }_1 + x^{ (t) }_2 + h^{ (t - 1) }_2 - 1.5 \\
			h^{ (t) }_{3} &=x^{ (t) }_1 + x^{ (t) }_2 + h^{ (t - 1) }_3 - 2.5 \\
			y^{ (t) } &= h^{ (t) }_{1} - h^{ (t) }_{2} + h^{ (t) }_{3} - 0.5
		\end{align*}
		
		This satisfies the Truth table:
		\begin{table}[h!]
			\centering
			\begin{tabular}{||c c  | c c | cc | cc | c||} 
				\hline
				$x_1$ & $x_2$ & $h^{(t - 1)}_1$ & $h^{(t)}_1$ & $h^{(t - 1)}_2$ & $h^{(t)}_2$ & $h^{(t - 1)}_3$ & $h^{(t)}_3$ & y\\ [0.5ex] 
				\hline\hline
				0 & 0 & 0 & 0 & 0 & 0 & 0 & 0 & 0 \\ \hline
				0 & 0 & 1 & 1 & 1 & 0 & 1 & 0 & 1 \\ \hline
				0 & 1 & 0 & 1 & 0 & 0 & 0 & 0 & 1 \\ \hline
				0 & 1 & 1 & 1 & 1 & 1 & 1 & 0 & 0 \\ \hline
				1 & 0 & 0 & 1 & 0 & 0 & 0 & 0 & 1 \\ \hline
				1 & 0 & 1 & 1 & 1 & 1 & 1 & 0 & 0 \\ \hline
				1 & 1 & 0 & 1 & 0 & 1 & 0 & 0 & 0 \\ \hline
				1 & 1 & 1 & 1 & 1 & 1 & 1 & 1 & 1 \\ [1ex] 
				\hline
			\end{tabular}
		\end{table}
		
	\end{homeworkProblem}
	\clearpage
	
	
	
	%---------------------------------------------------------------------------------
	%	PROBLEM 2
	%---------------------------------------------------------------------------------
	
	\begin{homeworkProblem}
		
		\begin{enumerate}
			\item 
			\begin{align*}
				\overline{h^{ (t) }} & = 
					1 + \overline{i^{ (t + 1) }} \dfrac{ \partial i^{ (t + 1) } }{ \partial h^{ (t + 1) } } + \overline{f^{ (t + 1) }} \dfrac{ \partial f^{ (t + 1) } }{ \partial h^{ (t + 1) } } + \overline{o^{ (t + 1) }} \dfrac{ \partial o^{ (t + 1) } }{ \partial h^{ (t + 1) } } + \overline{g^{ (t + 1) }} \dfrac{ \partial g^{ (t + 1) } }{ \partial h^{ (t + 1) } } \\
					&= 1 + \overline{i^{ (t + 1) }} \sigma^{-1}(w_{ix} x^{(t + 1)} + w_{ih} h^{(t + 1)}) w_{ih} 
					+  \overline{f^{ (t + 1) }} \sigma^{-1}(w_{fx} x^{(t + 1)} + w_{fh} h^{(t + 1)}) w_{fh}\\
					&\ \ \ 
					+ \overline{o^{ (t + 1) }} \sigma^{-1}(w_{ox} x^{(t + 1)} + w_{oh} h^{(t + 1)}) w_{oh} 
					+ \overline{g^{ (t + 1) }} \tanh^{-1}(w_{gx} x^{(t + 1)} + w_{gh} h^{(t + 1)}) w_{gh} \\
				\overline{c^{ (t) }} & = 
					\overline{h^{ (t) }} \dfrac{ \partial h^{ (t) } }{ \partial c^{ (t) } } 
					+ \overline{ c^{ (t + 1) }} \dfrac{\partial c^{ (t + 1) }}{\partial c^{ (t) }} \\
					&= \overline{h^{ (t) }} o^{ (t) } \tanh^{-1}(c^{(t)}) + \overline{c^{ (t + 1) }} f^{(t)}\\
				\overline{g^{ (t) }} & = 
					\overline{c^{ (t) }} \dfrac{ \partial c^{ (t) } }{ \partial g^{ (t) } } \\
					&= \overline{c^{ (t) }} i^{(t)}\\
				\overline{o^{ (t) }} & = 
					\overline{h^{ (t) }} \dfrac{ \partial h^{ (t) } }{ \partial o^{ (t) } } \\
					&= \overline{h^{ (t) }} \tanh(c^{(t)}) \\
				\overline{f^{ (t) }} & = 
					\overline{c^{ (t) }} \dfrac{ \partial c^{ (t) } }{ \partial f^{ (t) } } \\
					&= \overline{c^{ (t) }} c^{ (t - 1) } \\
				\overline{i^{ (t) }} &= 
					\overline{c^{ (t) }} g^{(t)}
			\end{align*}
			
			\item 
			\begin{align*}
				\overline{w_{ix}} &= \sum \limits_{t} \overline{i^{(t)}} \sigma^{-1}(w_{ix} x^{(t)} + w_{ih} h^{(t)}) x^{t} 
			\end{align*}
			
			\item
			This is because when $f^{(t)} = 1, i^{(t)} = 0$ and $o^{(t)} = 0$, 

			\begin{align*}
				\overline{c^{ (t) }} & = 
					\overline{h^{ (t) }} o^{ (t) } \tanh^{-1}(c^{(t)}) + \overline{c^{ (t + 1) }} f^{(t)}\\
					&= \overline{c^{ (t + 1) }}\\
				\overline{g^{ (t) }} & = 
					\overline{c^{ (t) }} i^{(t)}\\
					&= 0\\
				\overline{o^{ (t) }} & = 
					\overline{h^{ (t) }} \tanh(c^{(t)}) \\
				\overline{f^{ (t) }} & = 
					\overline{c^{ (t) }} c^{ (t - 1) } \\
				\overline{i^{ (t) }} & = 
					\overline{c^{ (t) }} \dfrac{ \partial c^{ (t) } }{ \partial i^{ (t) } } \\
					&= \overline{c^{ (t) }} g^{(t)}
			\end{align*}
			then $\overline{c^{ (t) }} = \overline{c^{ (t + 1) }}$ stays the same
			
			
		\end{enumerate}
		
	\end{homeworkProblem}
	\clearpage
	
	%----------------------------------------------------------------------------------------
	
\end{document}